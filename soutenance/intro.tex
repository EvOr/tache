\section*{Introduction}

\subsection*{Le contexte}
\frame
{
\frametitle{Pr\'esentation du contexte}
\framesubtitle{Internet et sa topologie}
\vfill
\begin{block}{Internet}
\begin{itemize}
\item Omnipr\'esent dans notre vie
\item  R\'eseau de systèmes autonomes (AS)
\end{itemize}
\end{block}
\vfill
\begin{block}{Typage}
\begin{itemize}
\item Contrats : ``clients à fournisseurs'' ou ``pairs à pairs''
\item AS : stubs ou transits
\end{itemize}
\end{block}
\vfill
\begin{block}{Topologie de l'Internet}
\begin{itemize}
\item Ensemble des liens logiques entre AS
\item Difficile de la repr\'esenter
\end{itemize}
\end{block}
\vfill
}

\subsection*{Les objectifs}
\frame
{
\frametitle{Pr\'esentation des objectifs}

\begin{block}{Objectifs de l'application}
\begin{itemize}
 \item Analyse de la topologie
 \item Affichage de la topologie
 \item Evolutivit\'e
\end{itemize}
\end{block}

\begin{block}{Objectifs p\'edagogiques}
\begin{itemize}
 \item Apprendre \`a utiliser la librairie Boost
 \item Approfondir notre connaissance de Qt
 \item Apprendre \`a utiliser git
\end{itemize}
\end{block}
}
