% \documentclass[a4paper,10pt]{article}

% \begin{document}

\subsection{Tomographie de l'Internet inter-domaine}

\subsubsection{Pr\'esentation d'Internet}

\par
Aujourd'hui, le r\'eseau le plus connu de part le monde est Internet. Il est le fruit combin\'e de la recherche militaire et de l'inter\^et que lui ont ensuite port\'e les universitaires.
\par
%TODO Déplacer en Accroche d'introduction doit pas être là ce truc... mais gg pala ça a la classe pour l'accroche.
Au d\'ebut utilis\'e pour relier entre eux les principaux sites militaires des \'Etats-Unis, puis les universit\'es, Internet relie aujourd'hui des millions de foyers de part le monde, mais que se cache-t-il r\'eellement derri\`ere cette appellation?
\par
%TODO Fin deplacer
Internet est en fait un r\'eseau de r\'eseaux. Des r\'eseaux ind\'ependants appartenant \`a des entreprises priv\'ees, des universit\'es ou à des op\'erateurs t\'el\'ephoniques. Ces r\'eseaux ind\'ependants sont appel\'es AS. Chaque r\'eseau est connect\'es \`a plusieurs autres et lorsque deux ordinateurs appartenant \`a des AS diff\'erents souhaitent communiquer, il faut router les donner \`a travers Internet. Aucun des \'el\'ements ne connait la totalit\'e de la topologie du r\'eseau et les paquets sont dirig\'es suivant des r\`egles de routage locales. Dans ces conditions, certains liens sont plus critiques que d'autres.
\par
Lorsque l'on parle de repr\'esenter Internet sous forme de graphe, il s'agit en fait de repr\'esenter les liens entre les diff\'erents AS. Cependant, avant de vouloir repr\'esenter ces liens encore faut-il les connaitre.

\subsubsection{Origine des donn\'ees}
\par
Il existe diff\'erents outils pour connaitre les liens entre les diff\'erents AS. Nous utilisons les donn\'ees du site \textit{www.caida.org} qui se basent sur trois outils pour obtenir leurs listes :
\begin{description}
 \item[traceroute : ] outil permettant de capturer les adresses IP des diff\'erents \'equipements r\'eseau sur un chemin entre une source et une destination en utilisant des paquets sonde UDP ou ICMP.
 \item[BGP : ] protocole de routage inter-domaine utilis\'e pour le routage entre les AS. Ses tables de routage contiennent des chemins d'AS, on peut ainsi avoir une idée des liens entre les diff\'erents AS.
 \item[WHOIS : ] collection de bases de donn\'ees contenant des informations utiles aux op\'erateurs t\'el\'ephoniques. Malheureusement, ces bases de donn\'ees sont maintenues \`a la main et ne  sont donc pas toujours tr\`es fiables. La plus sûre est la RIPE WHOIS qui regroupe des informations collect\'ees par la RIPE (service d'information sur les R\'eseaux IP Europ\'eens).
\end{description}
\par
Le croisement de ces diff\'erentes sources permet de faire correspondre des adresses IP avec des num\'eros d'AS et de savoir quels AS sont connect\'es avec quels autres et par quels types de liens.

\subsubsection{Internet IPv4 et IPv6}
\par
Aujourd'hui, un changement majeur qui est en train de s'op\'erer dans les r\'eseaux : le passage de l'IPv4 \`a l'IPv6. En effet, si IPv4 permet l'utilisation d'un peu plus de quatre milliards d'adresses ($2^{32}$), IPv6 permet quant \`a lui d'utiliser $2^{128}$ adresses diff\'erentes, ce qui permet de combler les nouveaux besoins.
\par
La mise en place progressive de l'IPv6 se traduit au sein des grands r\'eseaux par une cohabitation entre les deux syst\`emes. On peut ainsi trouver des informations pour cr\'eer le graphe de topologie pour l'internet IPv4 mais aussi pour l'internet IPv6.
\par
Au cours de notre projet, nous avons eu successivement acc\`es aux donn\'ees de l'IPv4 puis de l'IPv6.
% \end{document}
