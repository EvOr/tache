\par
Internet, le réseau des réseaux, est une interconnexion de systèmes autonomes, réseaux indépendants et raccordés entre eux. Compte tenu de sa taille, représenter Internet est une entreprise des plus difficile.

Le projet avait pour but de réaliser un outil permettant d'afficher et d'analyser la topologie de l'Internet. La finalité est de proposer un programme évolutif qui facilite l'implémentation de nouvelles méthodes d'affichage et d'analyse. Ces objectifs en cachent un plus pédagogique : maîtriser les outils à notre disposition et les utiliser lors du développement du logiciel.

Articulé autour du patron Modèle Vue Contrôleur, le logiciel est basé sur la librairie graphe de \boost qui réalise tous les traitements de construction et de manipulation sur la topologie. L'interface graphique, réalisée en Qt, est totalement indépendante de la partie traitement et permettra donc l'ajout de fonctionnalités facilement.

En cons\'equence, le travail de d\'eveloppement est parall\`elisable : une personne travaillant plus sur l'interface graphique, l'autre plus sur la couche de coeur. L'association des diff\'erentes parties est facilit\'ee par l'utilisation d'un gestionnaire de source : git.

L'affichage de la topologie a un rendu circulaire et est accompagné d'une fonction de zoom et d'un certain nombre de fonctions de filtrage et de calcul pour permettre à l'utilisateur, selon ses besoins, de se focaliser sur un aspect précis du graphe.

Finalement, l'outil r\'ealis\'e est \'evolutif, performant et son d\'eveloppement nous a permis d'approfondir nos connaissances dans les outils utilis\'es.\\

mots cl\'e : boost, Qt, interface graphique, évolutivité, topologie de l'Internet, git







