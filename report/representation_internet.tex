% \documentclass[a4paper,10pt]{article}

% \begin{document}

\subsection{Repr\'esentation de la topologie}

Repr\'esenter Internet sous la forme d'un graphe peut s'av\'erer particulièrement difficile. En effet, si l'on regarde les donn\'ees r\'ecolt\'ees au niveau de diff\'erentes sources, on s'aperçoit que le graphe \`a repr\'esenter est immense : pour la topologie IPv4, il comporte plus de 40000 sommets et un nombre tout aussi cons\'equent d'ar\`etes.
\par
La topologie en IPv6 est \`a peine plus petite. Cela vient du fait que cette technologie n'est pas encore aussi largement r\'epandue que l'IPv4.

\par
Au tout d\'ebut du projet, nous avons essayé de repr\'esenter le graphe complet, avec tous ses sommets et toutes ses ar\^etes. Le r\'esultat obtenu \'est, comme on peut s'y attendre, confu et illisible. Il nous a donc fallu envisager des solutions pour all\'eger nos traitements et notre affichage.
\par
Voici une capture d\'ecran du programme lorsqu'il repr\'esente le graphe dans son int\'egralit\'e :
%TODO insérer ici une capture d'écran avbec le graphe complet
%TODO faire une figure et la référencer dans le paragraphe
\par
Le principe est de nettoyer le graphe en enlevant les AS feuilles qui sont tr\`es nombreux et pas nécessairement tr\`es pertinents pour une \'etude de la topologie du coeur de l'Internet. 
%\par
\par
\'Etant donn\'e qu'Internet joue aujourd'hui un r\^ole important dans notre vie de tous les jours, tant au niveau professionnel que priv\'e, plusieurs \'etudes ont d\'ej\`a \'et\'e men\'ees pour mieux en comprendre la structure.
On trouve ainsi plusieurs ouvrage dans la litt\'erature et sur le net qui parle de ce sujet. On peut aussi trouver divers repr\'esentation graphique de l'Internet.
%TODO on pourrait ajouter cette image avec un commentaire, elle est sympa : http://en.wikipedia.org/wiki/File:Internet_map_1024.jpg

%\end{document}