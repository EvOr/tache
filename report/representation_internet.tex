% \documentclass[a4paper,10pt]{article}

% \begin{document}

\subsection{Repr\'esentation de la topologie}

Repr\'esenter l'Internet sous la forme d'un graphe peut s'av\'erer particulièrement difficile. En effet, si l'on regarde les donn\'ees r\'ecolt\'ees au niveau de diff\'erentes sources, on s'aperçoit que le graphe \`a repr\'esenter est immense : pour la topologie IPv4, il comporte 30079 sommets et 159368 ar\`etes.
\par
La topologie en IPv6 est compos\'ee quant\`a elle de 1308 sommets et 9420 sommets.

\par
Au tout d\'ebut du projet, nous avons essayé de repr\'esenter le graphe complet, avec tous ses sommets et toutes ses ar\^etes. Le r\'esultat obtenu \'est confu et illisible. Il nous a donc fallu envisager des solutions pour all\'eger nos traitements et notre affichage.
\par
Le principe est de nettoyer le graphe en enlevant les AS stubs qui sont tr\`es nombreux et pas nécessairement tr\`es pertinents pour une \'etude de la topologie du coeur de l'Internet. 
%\par
\par
\'Etant donn\'e qu'Internet joue aujourd'hui un r\^ole important dans notre vie de tous les jours, tant au niveau professionnel que priv\'e, plusieurs \'etudes ont d\'ej\`a \'et\'e men\'ees pour mieux en comprendre la structure.
On trouve ainsi plusieurs ouvrage dans la litt\'erature et sur le net qui parle de ce sujet. On peut aussi trouver divers repr\'esentation graphique de l'Internet, par exemple \textit{Lessons from three views of the internet topology}, \textit{caida.org}, \textit{wikipedia} lorsque l'on recherche le terme internet.

%\end{document}