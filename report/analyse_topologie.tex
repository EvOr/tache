%\documentclass[a4paper,12pt]{article}


%\begin{document}

\subsection{Analyse de la topologie}

\subsubsection{Les outils de la Th\'eorie des graphes}
\subparagraph{}
Pour analyser la topologie de l'Internet \`a partir du graphe le repr\'esentant, nous disposons de plusieurs outils fournis par la th\'eorie des graphes. Nous pr\'esentons ici cetaines caract\'eristiques int\'eressantes.

Par exemple, la recherche des cliques maximales (algorithme de Bron and Kerbosch) permet d'identifier les endroits o\`u les AS sont tous reli\'es deux \`a deux, un de ces ensembles est constitu\'e des op\'erateurs du \textit{Tier One}, sous r\'eserve de bien observer tous les liens les reliant dans les donn\'ees d'entr\'e. On retrouve aussi d'autres cliques localement.
\par
Ensuite, comme il a été expliqu\'e lors de la pr\'esentation de la topologie de l'Internet, le graphe a une structure connexe, c'est-\`a-dire que si l'on prend deux sommets quelconques, il existe un chemin entre eux.
Ce chemin passe par un certain nombre d'ar\^etes, et il est int\'eressant de savoir si telle ou telle ar\^ete est plus emprunt\'ee que d'autres.

\subsubsection{La centralit\'e}

\par
Prenons l'exemple d'un groupe d'AS A et B. Il existe au moins un chemin entre les diff\'erents AS de ces deux groupes. Supposons maintenant que tous ces chemins passent \`a un endroit du graphe par la m\^eme ar\^ete, celle-ci revet alors une importance capitale puisque l'ensemble des routes partant des membres du groupe A vers ceux du groupe B passent par l\`a.
\par
Il nous faut alors assigner des poids aux liens ou aux sommets afin de tenir compte de leur importance dans la structure globale.

\par
En th\'eorie des graphes, il existe une m\'etrique permettant d'\'evaluer l'importance des sommets ou des ar\^etes d'un graphe : la \textit{centralit\'e de Freeman}. Elle permet de calculer le ratio du nombre de chemins passant par cette ar\^ete ou ce sommet par rapport au nombre de chemins total dans le graphe. Plus ce ratio est important plus le noeud ou l'ar\^ete est vital pour garder le caract\`ere connexe du graphe. La centralit\'e se calcule gr\^ace \`a la formule suivante :

\begin{figure}[H]
   \centering
   \frame
   {
      \hspace{1em}
      \parbox{12cm}
      {
         \vspace{1em}Soit un graphe G=(V,E) avec n sommets, la centralité $c_b(v)$ d'une arête v est :
         \begin{equation}
               c_b(v) = \sum_{\underset{s \neq t}s \neq v \neq t \in V} \frac{\sigma_{st}(v)}{\sigma_{st}}
         \end{equation}
	o\`u $\sigma_{st}$ est le nombre de plus court chemin allant de s vers t et $\sigma_{st}(v)$ est le nombre de plus court chemin de s vers t passant par v.
      \vspace{1em}
      }\hspace{1em}
   }

% \end{tabular}
% \end{center}
  \caption{\label{centralit\'e}Calcul de la centralit\'e}
\end{figure}

% \begin{center}
%       \begin{tabular}{l}
%          \hline
%          \verb|sed -i -e "s/AS//g" nom_du_fichier|\\
%          \hline
%       \end{tabular}
%    \end{center}

%\end{document}
