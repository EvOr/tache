%\documentclass[a4paper,12pt]{article}


%\begin{document}

\subsection{Analyse de la topologie}

\par
Pour analyser la topologie de l'Internet \`a partir du graphe le repr\'esentant, nous disposons de plusieurs outils fournis par la th\'eorie des graphes.

\subsubsection{Les cliques}

Tout d'abord, la recherche des cliques permet d'identifier les endroits o\`u les AS sont tous reli\'es deux \`a deux, un de ces ensembles est constitu\'e des op\'erateurs du \textit{Tier One}. On retrouve aussi d'autres cliques localement.
%TODO peut etre a compléter

\subsubsection{La centralit\'e}

\par
Ensuite, comme il a été expliqu\'e lors de la pr\'esentation de la topologie de l'Internet, le graphe a une structure connexe, c'est-\`a-dire que si l'on prend deux sommets quelconques, il existe un chemin entre ces deux sommets.
Ce chemin passe par un certain nombre d'ar\^etes, et il est int\'eressant de savoir si telle ou telle ar\^ete est plus importante.
\par
Prenons l'exemple de deux AS A et B reli\'es entre eux. Nous savons qu'il existe une ar\^ete sur notre graphe entre A et B, supposons alors qu'il n'existe pas d'autres liens que celui entre A et B pour que les clients de A puissent joindre les clients de B. Ce lien revet donc une importance capitale puisque l'ensemble des routes partant des clients de A vers les clients de B passent par celui-ci.
\par
Il nous faut alors assigner des poids aux liens ou aux sommets afin de tenir compte de leur importance dans la structure globale.

\par
En th\'eorie des graphes, il existe un outil permettant d'\'evaluer l'importance des sommets ou des ar\^etes d'un graphe : la \textit{centralit\'e}. La centralit\'e permet de calculer le ratio du nombre de chemins passant par cette ar\^ete ou ce sommet par rapport au nombre de chemins total dans le graphe. Plus ce ratio est important plus le noeud ou l'ar\^ete est vital pour garder le caract\`ere connexe du graphe. La centralit\'e se calcule gr\^ace \`a la formule suivante :

\begin{figure}[H]
   \centering
   \frame
   {
      \parbox{12cm}
      {
         Soit un graphe G=(V,E) avec n sommets, la centralité $c_b(v)$ d'une arête v est :
         \begin{equation}
               c_b(v) = \sum_{\underset{s \neq v}s \neq v \neq t V} \frac{\sigma_st(v)}{\sigma_st}
         \end{equation}
	o\`u $\sigma_st$ est le nombre de plus court chemin allant de s vers t et $\sigma_st(v)$ est le nombre de plus court chemin de s vers t passant par v.
      }
   }
  \caption{\label{centralit\'e}Calcul de la centralit\'e}
\end{figure}



%\end{document}
