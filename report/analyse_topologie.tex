%\documentclass[a4paper,12pt]{article}


%\begin{document}

\subsection{Analyse de la topologie}

\par
Pour analyser la topologie de l'Internet \`a partir du graphe le repr\'esentant, nous disposons de plusieurs outils fournis par la th\'eorie des graphes.

\subsubsection{Les cliques}

Tout d'abord, la recherche des cliques permet d'identifier les endroits o\`u les AS sont reli\'es deux \` deux entre eux, un de ces ensembles est constitu\'e des op\'erateurs \textit{Tiers One}. On peut aussi retrouver localement d'autres cliques.
%TODO peut etre a compléter

\subsubsection{La centralit\'e}

\par
Ensuite, comme expliqu\'e dans la pr\'esentation de la topologie de l'Internet, le graphe a une structure connexe, c'est-\`a-dire que si on prend deux sommets quelconques, il existe un chemin entre ces deux sommets.
Ce chemin passe par un certain nombre d'ar\^etes, et il est int\'eressant de savoir si telle ou telle ar\^ete est plus importante.
\par
Prenons un exemple : deux AS A et B qui sont reli\'es entre eux. Il y a donc une ar\^ete sur notre graphe entre A et B. Supposons qu'il n'existe pas d'autre lien que celui entre A et B pour que les clients de A joignent les clients de B. Ce lien a donc une importance capitale car quand on prend l'ensemble des routes partant des clients de A vers les clients de B, elles passent par ce lien.
\par
Si on d\'ecide d'assigner des poids aux liens ou aux sommets, on souhaite alors tenir compte de leur importance dans la structure globale.

\par
En th\'eorie des graphes, il existe un outil permettant dans un graphe d'\'evaluer l'importance des sommets ou des ar\^etes, il s'agit de la \textit{centralit\'e}. La centralit\'e permet de calcul un ratio pour une ar\^ete ou pour un sommet, celui du nombre de chemins passant par cette ar\^ete ou ce sommet par rapport au nombre de chemin total dans le graphe. Plus ce ratio est important plus le noeud ou l'ar\^ete est vital pour garder le caract\`ere connexe du graphe. La centralit\'e se calcul gr\^ace \`a la formule suivante :

\begin{figure}[!ht]
   \centering
   \frame
   {
      \parbox{12cm}
      {
         Soit un graphe G=(V,E) avec n sommets, la centralité $c_b(v)$ d'une arête v est :
         \begin{equation}
               c_b(v) = \sum_{\underset{s \neq v}s \neq v \neq t V} \frac{\sigma_st(v)}{\sigma_st}
         \end{equation}
	o\`u $\sigma_st$ est le nombre de plus court chemin allant de s vers t et $\sigma_st(v)$ est le nombre de plus court chemin de s vers t passant par v.
      }
   }
  \caption{\label{centralit\'e}Calcul de la centralit\'e}
\end{figure}



%\end{document}
