\documentclass[12pt, a4paper, titlepage, oneside]{article}

% On commence par enjoliver
\usepackage[francais]{babel}	%typographie française
\usepackage[utf8]{inputenc}	% accents 8 bits dans le source
\usepackage[T1]{fontenc}	% accents dans le DVI
\usepackage{lmodern}	% Sans cette ligne la typo va être merdique!!!
\usepackage[usenames,dvipsnames]{color}
%\usepackage{nicebox}
% on inclue de l'utilitaire
\usepackage{float}
\usepackage{graphicx}
\usepackage{listings}
\newcommand{\Hrule}{\rule{\linewidth}{0.5mm}}

%On va se faire de belles marges
\usepackage{vmargin}
\setmarginsrb{1.5cm}{1.5cm}{1.5cm}{1.5cm}{15.71402pt}{1.5cm}{0.5cm}{1.5cm}

% de jolis headers/footers
\usepackage{fancyhdr}
\pagestyle{fancy}
\renewcommand{\sectionmark}[1]{\markright{\thesection\ #1}}
\renewcommand{\headrulewidth}{0.5pt}
\renewcommand{\footrulewidth}{0.5pt}
\fancyhead{} % clear all header fields
\fancyhead[L]{Implémentation de l'algorythme de Ricart et Agrawala}
\fancyhead[R]{\rightmark}
\fancyfoot{} % clear all footer fields
\fancyfoot[C]{\thepage}

%Gui : problème du absence conférence
%Gui : 

\begin{document}
	% on commence par la page de garde
	%\begin{changemargin}
\begin{titlepage}
	\begin{minipage}{0.5\textwidth}
		\begin{flushleft} \large
			%\includegraphics[bb=0 0 110 30]{images/logo-isima.png}\\
			\includegraphics[width=3.8cm]{./schema/logo-isima.jpg}\\
		\end{flushleft}
	\end{minipage}
	\begin{minipage}{0.43\textwidth}
		\begin{flushright} \large
			\includegraphics[width=3.8cm]{schema/logo-boost.png}
		\end{flushright}
	\end{minipage}

	\begin{minipage}{0.5\textwidth}
		\begin{flushleft} \large
			\textbf{I}nstitut \textbf{S}upérieur\\
			d'\textbf{I}nformatique de\\
			\textbf{M}odélisation et de\\
			leurs \textbf{A}pplications\\
			~\\
% 			\begin{small}
% 			Complexe des Cézeaux \\
% 			BP 125\\
% 			63173 Aubière Cedex
% 			\end{small}
		\end{flushleft}
	\end{minipage}
	\begin{minipage}{0.43\textwidth}
		\begin{flushright} \large
		\end{flushright}
	\end{minipage}


		\vfill
		\begin{center}
			\Hrule \\[0.4cm]
			\Large{Rapport de Projet de troisième année}\\		
			\Large Filière 5 : Infrastructure entreprise, réseaux et télécoms\\[1.0cm]
			\Huge{Représentation et analyse de la topologie internet}\\
			\Hrule\\[0.8cm]
\vfill 
% \textcolor{red}{\Huge{\textbf{MERCI JULIEN !}}}
% \vfill 
		\end{center}
		
		\vfill
		\begin{minipage}{0.5\textwidth}
			\begin{flushleft} \large
				\emph{Auteurs:}\\
				 Bastien \textsc{Legrand}\\
				 Jan \textsc{Villeminot}

			\end{flushleft}
		\end{minipage}
		\begin{minipage}{0.45\textwidth}
			\begin{flushright} \large
				\emph{Tuteur de projet:} \\
				Mickael \textsc{Meulle}\\
			\end{flushright}
		\end{minipage}
		
		\vfil
		\begin{center}
			{\large 2008-2009}
		\end{center}
	\end{titlepage}
%\end{changemargin}
	%\par
Internet, le réseau des réseaux, est une interconnexion de systèmes autonomes, réseaux indépendents et raccordés entre eux. Compte tenu de sa taille, représenter Internet est une entreprise des plus difficiles.

Le projet avait pour but de réaliser un outil permettant d'afficher et d'analyser la topologie de l'Internet. La finalité est de proposer un programme évolutif qui facilite l'implémentation de nouvelles méthodes d'affichage et d'analyse. Ses objectifs en cachent un plus pédagogique : maîtriser les outils à notre disposition et les utiliser lors du développement du logiciel.

Articulé autour du pattern Modèle Vue Contrôleur, le logiciel est basé sur la librairie graphe de \boost qui réalise tous les traitements de construction et de manipulation sur la topologie. L'interface graphique, réalisée en Qt est totalement indépendante de la partie traitement et permettera donc l'ajout de fonctionnalités facilement.

L'affichage de la topologie a un rendu circulaire et est accompagné d'une fonction de zoom et d'un certain nombre de fonctions de filtrage et de calcul pour permettre à l'utilisateur, selon ses besoins, de se focaliser sur un aspect précis du graphef.

mots cl\'e : boost, Qt, interface graphique, évolutivité, topologie de l'Internet








	%\section*{Remerciements}

Nous tenons \`a remercier tout particuli\`erement Monsieur Mickael Meulle, notre tuteur de projet, pour son accompagnement tout au long du parcours et pour sa disponibilit\'e malgr\'e la distance.	
	% puis vient la table des matières
	%\pagenumbering{roman}
%	\listoffigures
%\pagebreak
 %	\tableofcontents
%	\pagebreak\textit{\textit{}}
%\fancyhead[R]{Lexique}
	%\input{Lexique.tex}
 %	\pagebreak
        
	%\fancyhead[R]{Introduction}
	%\section*{Introduction}

\subsection*{Le contexte}
\frame
{
\frametitle{Pr\'esentation du contexte}
\framesubtitle{Internet et sa topologie}
\vfill
\begin{block}{Internet}
\begin{itemize}
\item Omnipr\'esent dans notre vie
\item  R\'eseau de systèmes autonomes (AS)
\end{itemize}
\end{block}
\vfill
\begin{block}{Typage}
\begin{itemize}
\item Contrats : ``clients à fournisseurs'' ou ``pairs à pairs''
\item AS : stubs ou transits
\end{itemize}
\end{block}
\vfill
\begin{block}{Topologie d'Internet}
\begin{itemize}
\item Ensemble des liens logiques entre AS
\item Difficile de la repr\'esenter
\end{itemize}
\end{block}
\vfill
}

\subsection*{Les objectifs}
\frame
{
\frametitle{Pr\'esentation des objectifs}
Objectifs de l'application :
\begin{itemize}
 \item Analyse de la topologie
 \item Affichage de la topologie
 \item Evolutivit\'e
\end{itemize}

Objectifs p\'edagogiques :
\begin{itemize}
 \item apprendre \`a utiliser la librairie Boost
 \item approfondir notre connaissance de Qt
 \item apprendre \`a utiliser git
\end{itemize}

}

	\pagebreak 	
% 	% puis la page d'intro
\fancyhead[L]{Systemes Repartis}
\fancyhead[R]{\rightmark}
	
% 	\fancyhead{}
% 	\fancyhead[R]{Rapport de Design Électronique}
% 	\section*{Introduction}

\subsection*{Le contexte}
\frame
{
\frametitle{Pr\'esentation du contexte}
\framesubtitle{Internet et sa topologie}
\vfill
\begin{block}{Internet}
\begin{itemize}
\item Omnipr\'esent dans notre vie
\item  R\'eseau de systèmes autonomes (AS)
\end{itemize}
\end{block}
\vfill
\begin{block}{Typage}
\begin{itemize}
\item Contrats : ``clients à fournisseurs'' ou ``pairs à pairs''
\item AS : stubs ou transits
\end{itemize}
\end{block}
\vfill
\begin{block}{Topologie d'Internet}
\begin{itemize}
\item Ensemble des liens logiques entre AS
\item Difficile de la repr\'esenter
\end{itemize}
\end{block}
\vfill
}

\subsection*{Les objectifs}
\frame
{
\frametitle{Pr\'esentation des objectifs}
Objectifs de l'application :
\begin{itemize}
 \item Analyse de la topologie
 \item Affichage de la topologie
 \item Evolutivit\'e
\end{itemize}

Objectifs p\'edagogiques :
\begin{itemize}
 \item apprendre \`a utiliser la librairie Boost
 \item approfondir notre connaissance de Qt
 \item apprendre \`a utiliser git
\end{itemize}

}

% 	\pagebreak
	\section{Les donn\'ees, les outils}

\subsection{Les donn\'ees}
\frame
{
\frametitle{Les donn\'ees}


}

\subsection{Outils d'analyse}
\frame
{
\frametitle{Outils d'analyse}


}

\subsection{Outils de d\'eveloppement}
\frame
{
\frametitle{Outils de d\'eveloppement}


}
\pagebreak
\section{Le logiciel de visualisation et d'analyse de l'internet}
\subsection{Objectifs}
Le but du projet est de produire un logiciel qui permette la visualisation de la topologie d'internet et en particulier du coeur du réseau.

\subsubsection{Objectifs de l'application}
\label{obj}
\paragraph{Analyse de la topologie}
Construire la topologie de l'Internet et l'analyser grâce à différents algorithmes. La librairie \textit{Boost} servant de base pour ces derniers.

\paragraph{Affichage de la topologie}
Réaliser une interface agréable dans la librairie de notre choix et lui permettre d'afficher l'intégralité de l'analyse réalisée.

\paragraph{Evolutivité}
Permettre l'implémentation de nouveau algorithmes d'analyse et de placement de points facilement.

\subsubsection{Objectif pédagogiques}
Au delà de l'objectif de réalisation d'une application et de l'organisation que cela nécessite, un certain nombre d'objectifs pédagogique ont été convenus.

\paragraph{Objectifs fixés dès le départ : } 
\paragraph{} L'objectif principal est bien évidemment d'apprendre à maîtriser la librairie boost (C++), principalement sa partie sur les graphs. Il était aussi question d'approfondir notre connaissance de la topologie de l'Internet.

\paragraph{Objectifs d'auto-apprentissage : }
\paragraph{}  L'utilisation d'une librairie graphique pour représenter la topologie de l'Internet était nécessaire. Son choix nous était réservé. Nous avons choisi d'utiliser Qt pour deux raisons. La première est que nous voulions approfondir le cours sur Qt que nous allions avoir dans le cadre de nos études à l'ISIMA. La seconde est que, sous la pression de Nokia, cette librairie est récemment passée sous licence LGPL ce qui veut dire que les développeurs d'applications commerciales vont pouvoir développer gratuitement autour de Qt (sans payer les très onéreuses licences jusqu'à présent nécessaires pour vendre quelque chose). Il est fort probable que Qt prennent beaucoup plus d'ampleur dans les prochaines années.

\paragraph{} Nous nous sommes fixés un autre objectif : celui d'apprendre à utiliser git. Pour les raisons évoquées en \ref{gitPar}, subversion ne nous convenait pas, mais nous ne connaissions pas vraiment git pour autant. Après avoir visionner une conférence de Linus Torvalds sur le sujet, il nous a semblé logique et nécessaire d'appréhender git et de s'en servir en particulier pour notre projet. D'autant plus que l'un comme l'autre nous avions effectué nos stages avec subversion ou cvs.

\subsubsection{Déroulement du projet}

La quasi totalité du projet a été mené par des échanges de mails, et des discussions sur des messageries instantannées (Skype, gchat). De plus, il nous a donné une grande autonomie sur les décisions à prendre pour mener à bien nos objectifs. Ce mode de communication nous a permis d'avoir plus d'expérience dans le travail à distance.

L'autonomie demande une bonne organisation. En effet, quelques soient les décisions ou les problèmes rencontrés, il faut en rendre compte au tuteur lors de la prochaine réunion de travail suivante, réunion qui avait régulièrement lieu le Jeudi après-midi.

\subsubsection{Méthode de travail}

Afin d'optimiser la production, nous avons adopté une méthode de développement par itération. En fait, nous nous sommes répartis les tâches selon des fonctionnalités ``élémentaires''. Après avoir isolé l'ensemble des fonctionnalités nécessaires d'une semaine à l'autre, nous ré-affectons des priorités aux différentes tâches afin d'être plus efficace dans leur parallèlisation et leur ordonnancement.

\subsubsection{La librairie graph de boost}
%\documentclass[a4paper,12pt]{article}

%\begin{document}

\subsection{La librairie \boost}

\par
La \boost est un ensemble de biblioth\`eques logicielles \'ecrites en C++ pour am\'eliorer les fonctionnalit\'es de celui-ci. La \boost est compos\'ee de nombreux modules dont voici quelques exemples parmis lesquels on retrouve des fonctionnalit\'es de bases comme les entr\'es / sorties, la gestion des erreurs, et aussi d'autres plus avanc\'ees comme des fonctions de gestions de graphes, de programmation lin\'eaire, de g\'en\'eration de nombres pseudo-al\'atoires, de traitement d'images, de multithreading, et bien d'autre encore.
\par
L'\'ecriture de nouveaux \'el\'ements pour la \boost est soumis \`a un  comit\'e de lecture et l'ensemble des biblioth\`eques est plac\'e sous une licence sp\'eciale : la \textit{Boost Software Licence} qui permet d'inclure des modules de \boost aussi bien dans des projets Open Source que dans des projets propri\'etaires.
\par
\'Etant donn\'e que nombre des membres du comit\'e de lecture de la \boost font partie du comit\'e du standard C++, plusieurs biblioth\`eques issues de la \boost ont \'et\'e s\'electionn\'ees pour faire partie de la prochaine norme du C++.
\par
Au cours de notre projet nous avons beaucoup utilis\'e la partie gestion de graphes de \boost ainsi que certaines des fonctions d'entr\'es / sorties am\'elior\'ees.

%\end{document}



\subsubsection{La librairie graphique Qt}
\par
Qt, prononcer ``cute'' de l'anglais joli, est une biblioth\`eque logicielle orient\'ee objet et d\'evelopp\'ee en C++ par la soci\'et\'e \textit{Qt Software}, anciennement connue sous le nom de \textit{Trolltech}. Elle offre des composants graphique permettant entre autre l'acc\`s aux donn\'ees, les connexions r\'eseaux, la gestion des files d'ex\'ecution et pleins d'autres choses encore. Cette biblioth\`eque est aujourd'hui dans sa version quatre.
\subparagraph{}
Qt permet la portabilit\'e des applications par simple recompilation du code source. Elle est support\'ee sous des environnements tels que Unix/Linux, Windows et Mac OS X.
\subparagraph{}
Qt est principalement connue pour \^etre la biblioth\`eque sur laquelle repose l'environnement graphique KDE, l'un des environnements de bureau les plus utilis\'e dans le monde Linux. Un certains nombres d'applications tr\`es connues du grand public et recens\'ees sur le site de \textit{Trolltech} utilisent Qt : VLC Media Player, Skype, Google Earth...

\par
L'outil Qt repose sur trois parties essentielles : Qt Designer, les slots et signaux, et le compilateur de meta-objet MOC.
\subparagraph{}
Qt Designer est un outil graphique permettant de créer facilement des interfaces graphiques avec Qt, d'y ajouter des panneaux et des boutons et d'y connecter des fonctions sous la forme de signaux et de slots. Au cours de notre projet, nous n'avons que tr\`es peu utilis\'e cet outil, pr\'ef\'erant explorer les \'etapes de cr\'eation d'une interface graphique en r\'ealisant nous m\^eme tout le code.
\subparagraph{}
Les slots et signaux sont les m\'ecanismes sp\'ecifiques \`a Qt qui lui permettent de g\'erer les \'echanges d'informations entre les diff\'erents \'el\'ements d'une interface graphique. Il s'agit en fait d'une impl\'ementation du patron de conception observateur.
Les objets Qt poss\`edent des slots et des signaux. Un objets peut \'emettre des signaux qui seront re\c cus par les slots d'un autre objet. Ainsi, si on veut cr\'eer un bouton quitter dans une fen\^etre, pour le rendre fonctionnel, il faut connecter le signal correspondant à l'action \textit{je clique sur le bouton} au slot \textit{la fen\^etre se ferme}.
\subparagraph{}
Le compilateur de m\'eta-objets MOC (pour meta object compiler) est un pr\'eprocesseur qu iest appliqu\'e au code source de l'application avant sa compilation et qui r\'eunit les informations n\'ecessaires au fonctionnement des slots et signaux et \'eventuellement \`a l'introspection. L'introspection est la capacit\'e d'un programme \`a connaitre son \'etat, examiner ses structures internes et \'eventuellement modifier des objets le composant au cours de l'\'execution. Cela va de pair avec les slots et signaux, puisque pour cacher une fen\^etre, on change sons \'etat de visible \`a non visible.


\subsection{Méthodologie de développement et redistribution}
\subsubsection{Documentation avec Doxygen}
\par
Doxygen est un logiciel libre de documentation automatique de code. Il a \'et\'e en grande partie \'ecrit par Dimitri van Heesch. Son nom est la contraction de \textit{dox} (pour docs, de l'anglais documents) et \textit{gen} (de l'anglais generator). Il est capable de g\'en\'erer de la documentation pour des codes \'ecrits en C, C++, Java, Python et d'autres langages encore. Il s'appuie pour cela sur un certains nombre de commentaires \'ecrits dans le code avec une syntaxe sp\'eciale. La documenntation g\'en\'er\'ee peut \^etre en HTML, ou en \LaTeX.
\par
Doxygen extrait en particulier les prototypes et documentations des fonctions, les fichiers inclus, la documentation des structures de donn\'ees, les prototypes et la documentation des classes. Il peut repr\'esenter leur hi\'erarchie sous forme de sch\'ema UML. Il garde un index de tous les idenntifiants, et permet une navigation avec HTML dans les fichiers sources comment\'es.
\par
Voici un petit exemple d'utilisation de la syntaxe Doxygen pour documenter une fonction :
\begin{figure}[ht]
\centering
\frame{
\parbox{16cm}{
   $/// \backslash \textrm{brief Ajoute une arette} \\
   /// \backslash \textrm{param i1 index du premier point} \\
   /// \backslash \textrm{param i2 index du deuxieme point} \\
   /// \backslash \textrm{param linkType descripteur du type d'arrete} \\
   /// \backslash \textrm{param found booleen resultat} \\
   /// \backslash \textrm{param e edge\_descriptor de l'arrete} \\
   /// \backslash \textrm{param g double pointeur sur le Graph ou il faut ajouter la     relation} \\
   /// \backslash \textrm{return le type de point}$
}
}
\caption{\label{exemple_doxygen} Exemple de syntaxe Doxygen}
\end{figure}

On peut y voir une br\`eve description du r\^ole de la fonction, des informations sur ses diff\'erents param\`etre et son type de retour.

\subsubsection{Redistribution et travail en équipe}
\paragraph{subversion : des limites}
%TODO supprimer premier paragraphe
%EvOr se défoule
%La plupart des personnes ayant utilisé subversion de manière professionnelle en sont conscientes il est loin d'être agréable à utiliser, sa gestion des branches est désastreuses, et les conflits sont gérés de manière abruptes et souvent inefficaces, pour palier à ses défauts les entreprises mettent en place des politiques de fonctionnement autour du gestionnaire de version drastique avec par exemple interdiction de commiter ce qui ne marche pas. Au delà de ses problèmes de fonctionnement subversion est lent, très lent et nécessite en plus la présenced'un serveur pour pouvoir commiter.
%Fin EvOr se défoule

\paragraph{} L'une des bases du développement est d'utiliser un gestionnaire de sources afin de pouvoir modifier ses fichiers sans se soucier des conséquences. La première chose que nous avons faite, avant même de passer au développement, est de mettre en place un serveur subversion sur l'une de nos machines. Le problème qui s'est très vite posé est le suivant : quelque soit l'endroit où l'autre travaille, s'il n'a pas accès au serveur, il ne peut pas l'utiliser (\verb|commit/checkout|) et doit donc développer sans gestionnaire de sources (figure \ref{svn})... Il nous fallait un moyen de contrôler nos sources de manière distribuée : git.

\begin{figure}[H]
\begin{center}
        \includegraphics[width=0.65\textwidth]{./schema/svn.png}
\caption{La gestion classique des sources : centralisation totale }
\label{svn}
\end{center}
\end{figure}


\paragraph{git : Le gestionnaire de source distribué}
\label{gitPar}
\paragraph{Histoire :}
\subparagraph{} Git a été créé par Linus Torvalds en 2004, il avait besoin de remplacer BitKeeper pour la gestion des sources du noyau Linux, et il est évident qu'il ne peut pas mettre en place un serveur subversion pour un projet aussi vaste et avec autant de développeurs répartis au quatre coins du monde. Il a donc écrit, en quelques jours, un gestionnaire de sources distribué performant et qui a pour objectif de ne pas géner le développeur. En outre, Git effectue un checksum SHA1 des sources à chaque fois qu'une modifications est effectuée (commit ou push) ce qui permet d'être sûr de l'unicité des données...

\paragraph{Que veut dire distribué ?} 
\subparagraph{}Git est distribué, cela veut dire que chaque machine l'utilisant est à la fois un serveur et un client. En d'autres termes, chaque développeur peut ``commiter'' et effectuer des \verb|revert| en local. Ce qui permet de développer de manière beaucoup plus efficace, puisque les commits sont effectués en quelques millisecondes (le temps d'afficher la trace). Une fois que le développeur est satisfait de son travail, il peut demander aux autres de récupérer les sources (\verb|pull|)  depuis sa machine, ou il peut  envoyer les sources (\verb|push|) sur un serveur de fichiers central par exemple (figure \ref{git}). Tout ce que nécessite git pour fonctionner est un client et/ou un serveur ssh.


\begin{figure}[H]
\begin{center}
        \includegraphics[width=0.8\textwidth]{./schema/git.png}
\caption{La gestion des sources avec git : Distribution et liberté}
\label{git}
\end{center}
\end{figure}

\paragraph{Fonctionnement :} 
\subparagraph{}La comparaison de fichiers sous git est un peu particulière, contrairement à subversion qui se base sur le nom du fichier et le numéro du commit, git se base sur un checksum du fichier. Il se sert en fait d'une base interne contenant l'intégralité des checksums des fichiers en fonction du numéro du commit correspondant mais ne se base que sur le checksum pour vérifier si deux fichiers sont différents, l'opération est donc très rapide et sûr.

\subparagraph{} Concrètement, la clé du fonctionnement de git est l'utilisation massive des branches. En effet, dans git tout est une branche : chaque machine de chaque développeur est une branche et ce de manière totalement transparente. C'est à dire que le développeur peut \verb|commit| et \verb|revert| sa propre branche et l'envoyer ensuite sur la branche principale \verb|push| (\verb|master|). Avec les sources, l'historique des commits effectués en local est envoyé et n'importe quel développeur peut annuler n'importe quel commit...

\subparagraph{} Il est donc très facile de créer, de merger et de supprimer une branche avec git, tellement facile qu'on se demande comment ça peut être aussi difficile avec subversion.

\subparagraph{} En cas de conflit, git ne fait rien ; pour git le simple fait que deux fichiers devant être ``mergés'' soient plus ``récents'' que leur dernière version commune est un conflit. Il signale alors à l'utilisateur qu'il faut ``merger''. Et ce merge est très aisé, il se déroule automatiquement la plupart du temps, à la main parfois : git est capable de ne montrer uniquement les parties de fichiers en conflit.


\subparagraph{}Nous vous renvoyons à notre annexe sur git pour un descriptif détaillé des commandes à connaître avec git.

%TODO en annexe les commandes de git
%Git fonctionne en ligne de commande, néanmoins ses développeur l'ont doté d'une interface graphique plutôt fonctionnelle. Git possède un nombre impressionant de fonctionnalités, nous allons toutefois lister celles qui nous semblemt les plus importantes.

\paragraph{Et l'accès aux données depuis l'extérieur ?}
\subparagraph{} A l'instar de sourceforge pour subversion, il existe quelques sites communautaires permettant de déposer les sources de ses projets sur des serveurs distants, github en fait partie. En choisissant git, nous voulions, outre la puissance et la distribution, avoir accès aux modifications de l'un et de l'autre sans avoir besoin de s'appeler pour se demander d'allumer nos machines.

Centraliser nos sources sur la toile, nous a semblé être la meilleure solution. Github fonctionne avec git, il lui suffit d'avoir votre clé publique rsa et il est possible d'utiliser git avec le site. 

\paragraph{En résumé}

\subparagraph{}L'utilisation conjuguée de git et de github nous permet :
\begin{itemize}
 \item De développer avec un gestionnaire de source sans accès au réseau ;
 \item D'avoir accès à notre projet par une simple commande quelque soit la machine sur laquelle nous travaillons ;
\item De développer avec un outil puissant, rapide et efficace.
\end{itemize}

\subsection{Aperçu du logiciel}
Après avoir vu les outils et les méthodes utilisés pour développer le logiciel, passons à la présentation de l'application à proprement parler.

\subsubsection{Modèle Vue Controlleur}
\label{mvcText}
Lorsque l'on développe une application disposant d'une interface graphique, la base du développement est de séparer la partie traitement ou métier de l'interface à proprement parler. Il est souvent utile de rajouter une couche d'interfaçage entre la partie métier et la partie affichage. Cette modélisation s'appelle Modèle Vue Controlleur. Nous avons choisi d'implémenter ce modèle dans notre projet. La partie métier étant le graphe représentant les AS de l'internet et leur liens. Un schéma très superficiel du pattern est présenté en figure \ref{mvc}.

L'utilisation que nous avons faites de l'objet Graph une fois créé, analysé et rempli, peut être assimiler à l'accès à une base de données. L'ensemble des méthodes d'accès se trouvant dans le controlleur, la partie métier n'en a absolument aucune connaissance. De même, la partie affichage n'a accès qu'aux méthodes du contrôleur.

Le but de ce pattern est de permettre aux développeurs de changer la vue en ne modifiant aucune autre partie du code, ou de modifier la partie métier sans modifier le code de la vue. Ce modèle met l'accent sur l'évolutivité et la maintenabilité du code. 

En outre, un tel découpage facilite la répartition des tâches dans l'équipe.

\begin{figure}[H]
\begin{center}
        \includegraphics[height=0.3\textheight]{./schema/mvcScheme.png}
\caption{Schéma simplifié du Modèle Vue Controlleur}
\label{mvc}
\end{center}
\end{figure}

\subsubsection{La base du développement : l'objet Graph}

Pour manipuler, une structure de graphe en mémoire nous avons défini un objet \verb|Graph| à partir des templates exisants dans \textit{Boost}. Notre object se repose sur une liste d'ajacence de la librairie \textit{Boost} que nous avons adapté les données stockées par les noeuds et les arcs.

Les éléments du graphe sont identifiés par des descripteurs. Pour les sommets : \verb|vertex_descriptor|, pour les arêtes : \verb|edge_descriptor|. Bien que ces descripteurs soient des simples entiers, il ne nous était pas possible d'utiliser les numéros des AS. Chaque numéro d'AS est stocké pour chaque descripteurdans les données associés aux noeuds de la classe \verb|Graph|. %Dans le cas général ces descripteurs ne sont pas des entiers, néanmoins, afin de 

Notre utilisation de l'objet \verb|Graph| ressemble à celui que l'on ferait d'une base de données. Par analogie, nous possèderions une table de sommets, et une table d'arêtes. Chacune de ces tables conservent un ensemble d'attributs relatifs à chacun de ses éléments. La clé primaire de chacune de ces tables serait le descripteur correspondant (\verb|edge_descriptor| pour les arcs et \verb|vertex_descriptor| pour les noeuds). Les champs des tables corspondent aux atttributs des classes AS et ASLink (figure \ref{bdd}).
Ces objets possèdent un certain nombre d'attributs nécessaires au bon fonctionnement de boost en plus de ceux qui nous intéressent : le type et le numéro de l'AS. Ils sont accessibles de la manière suivant : \verb|graph[descriptor].type|.


\begin{figure}[H]
\begin{center}
        \includegraphics[width=0.8\textwidth]{./schema/bdd.png}
\caption{Représentation des types d'objet du graphe}
\label{bdd}
\end{center}
\end{figure}

\subsubsection{Représentation idéale}
La figure 8 montre un exemple de représentation de l'Internet. Le but de notre logiciel est d'obtenir une représentation de ce type qui permet de distinguer la structure d'interconnexion de l'Internet.

 \ref{ideal}. 
\begin{figure}[H]
\begin{center}
        \includegraphics[width=0.8\textwidth]{./schema/Internet_map_1024_transparent.png}
\caption{Exemple de représentation de l'Internet [ source wikipedia ] }
\label{ideal}
\end{center}
\end{figure}


\pagebreak
\section{Utilisation du logiciel}

Au cours de l'avancement du projet nous avons choisi un juste milieu entr ela quantit\'e de d\'eveloppementde l'interface graphique et les fonctionnalit\'es d'analyse du graphe. Ambitieux, nous souhaitions pousser l'analyse loin et réaliser une interface graphique évoluée.

\subsection{Fonctionnement global}
Le logiciel fonctionne \`a partir de fichiers de données recensant la topologie d'internet et sa structure. Il doit permettre entre autre de construire un graphe et de permettre son affichage dans une fenêtre en Qt.

Rappelons le voeu de l'application est d'utiliser les structures de donn\'ees \textit{Boost} et que nous avons adopté le pattern MVC afin de développer de manière plus ind\'ependante l'interface graphique du coeur de l'application. La coeur de ce système est la communication entre le modèle et la vue, en d'autres termes : le contrôleur. Voici la liste de ses fonctionnalités :
\begin{itemize}
 \item lancer la lecture des fichiers de données afin de construire le graphe,
\item obtenir le nombre d'AS, de liens, et d'autres informations globales sur le graphe,
\item de récupérer le graphe sans les stubs,
\item de récupérer l'adjacence d'un AS,
\item de récupérer les informations concernant un AS (centrality, numéro),
\item de ne récupérer qu'une partie du graphe en fonction de la centralité des sommets,
\item de calculer les coordonnées que les sommets doivent avoir sur l'interface graphique.
\end{itemize}

L'ensemble de ses fonctionnalités se retrouve dans l'interface graphique, qui sera présentée ci-dessous.

%ATTENTION : \'ebauche pour cette partie.
\subsection{Description du programme}

Le programme tel que le voit l'utilisateur est une fen\^etre graphique avec un menu en haut, une zone d'affichage au centre, et une zone de notification en bas.

\begin{figure}[H]
\centering
 \fbox
 {
 \includegraphics[width=16cm]{./schema/capture_ecran_programme.png}
 }
  \caption{\label{ecran_principal}Ecran principal du programme}
\end{figure}


Le menu comporte trois types d'entr\'ees :
\begin{description}
 \item[Fichier] permet l'ouverture des fichiers de donn\'ees \`a lire ou la fermeture du programme,
 \item[Options] permet les interactions avec le graphe telles que l'effacement de la zone d'affichage ou encore les recherche d'informations sur un AS,
 \item[About] permet l'affichage d'informations sur le programme.
\end{description}

\par
De nombreuses options sont disponibles pour l'utilisateur, et selon celles qu'il choisira d'utiliser, il en d\'ebloquera d'autres. Ces options permettent de jouer sur l'affichage du graphe et d'obtenir des informations sur les AS. Les fonctionnalit\'es impl\'ement\'ees sont les suivantes :
\begin{itemize}
 \item R\'ecup\'eration d'informations sur un AS,
 \item Calcul du nombre de cliques maximum dans le graphe gr\^ace \`a l'algorithme de Bron and Kerbosch,
 \item Chargement d'un fichier de triplets pour \'eliminer les stubs du graphe,
 \item Zoomer sur le proche voisinage d'un AS,
 \item Revenir au graphe de base,
 \item Calculer la centralit\'e de Freeman des sommets,
 \item Afficher seulement les sommets avec une centralit\'e sup\'erieure \`a une certaine valeur,
 \item Afficher seulement les sommets avec une centralit\'e inf\'erieure \`a une certaine valeur
 \item Afficher seulement les sommets avec un num\'ero d'AS sup\'erieur \`a une certaine valeur,
 \item Afficher seulement les sommets avec un num\'ero d'AS inf\'erieur \`a une certaine valeur
 \item effacer le graphe pour commencer une nouvelle \'etude.
\end{itemize}

Ces fonctions sont d\'ebloqu\'ees comme suit :

\begin{figure}[H]
\centering
 \includegraphics[width=0.8\textwidth]{./schema/seqMenu.png}
  \caption{\label{seq_option}S\'equence des d\'ev\'erouillage d'options}
\end{figure}


\subsection{L'exp\'erience utilisateur}
\par
Lors du lancement du programme, l'utilisateur se retrouve devant un fen\^etre o\`u la zone d'affichage est vide et les compteurs de la zone de notification sont tous \`a z\'ero comme c'est le cas sur la figure \ref{ecran_principal}.
\par
\`A ce stade, l'utilisation des options est inutile car elles sont toutes bloqu\'ees en attendant qu'un fichier soit charg\'e prooduisant un nouveau graphe. L'action du menu \textit{About} permet d'avoir des informations sur le programme, comme montr\'e figure \ref{ecran_about}.

\begin{figure}[H]
\centering
 \fbox
 {
 \includegraphics[width=8cm]{./schema/capture_ecran_about.png}
 }
  \caption{\label{ecran_about}Fen\^etre d'informations sur le programme}
\end{figure}

\par
Dans le menu \textit{Fichier}, l'utilisateur a le choix entre ouvrir un fichier de donn\'ees pour construire un graphe ou quitter le programme. Ce menu est illustr\'e figure \ref{ecran_fichier}.

\begin{figure}[H]
\centering
 \fbox
 {
 \includegraphics[width=12cm]{./schema/capture_ecran_fichier.png}
 }
  \caption{\label{ecran_fichier}Menu fichier}
\end{figure}

Lorsque l'utilisateur choisit d'ouvrir un fichier de donn\'ees, une nouvelle fen\^etre de navigation s'ouvre et lui demande de choisir son fichier. Il choisit un fichier de donn\'ees \`a ouvrir et le logiciel se charge d'afficher le graphe correspondant en organisant les sommets sur un polyg\^one r\'egulier \`a n c\^ot\'es.
La barre de statut est mise \`a jour avec des informations telles que le nombre de sommets, le nombre d'ar\^etes ou le temps de calcul. Un exemple de graphe est donn\'e figure \ref{ecran_graph}

\begin{figure}[H]
\centering
 \fbox
 {
 \includegraphics[width=16cm]{./schema/capture_ecran_graph.png}
 }
  \caption{\label{ecran_graph}Ecran principal du programme}
\end{figure}

\par
Ensuite, les options d\'ebloqu\'ees lui permettent d'int\'eragir avec le graphe.

\subsection{Mise en évidence des caractéristiques d'internet}
\par
Le programme \'eclaire de fa\c con visuelle certains aspects \'evoqu\'es plus t\^ot dans ce rapport. En effet, on voit au premier chargement d'un fichier que le graphe ressemble \`a un gros ovale bleu, c'est d\^u au grand nombre d'As qui compose internet et qui rend la repr\'esentation difficile. En comparaison, la structure en IPv6 est clairement plus l\'eg\`ere puisqu'on peut voir du blanc.
\par
Une fois \'elimin\'es les stubs qui ne sont pas pertinents dans le cadre d'une \'etude de topologie du coeur, on peut voir que le nombre de clique et on s'apper\c coit qu'en IPv6, il n'y en a qu'une.
\par
Enfin, le calcul de la centralit\'e peut permettre de visualiser plus facilement les ar\^etes sensibles m\^eme si le programme ne dispose pas enccore de mode d'affichage sp\'ecifique pour mettre ce point en valeur.
\par
Dans le futur, le logiciel pourra accueillier des ``extensions'' permettant de réaliser de nouveaux traitements sur l'affichage ou les données.
%\pagebreak
%\fancyhead[R]{Conclusion}
	%\section*{Conclusion}

\frame
{
\frametitle{Conclusion}


}
%\pagebreak
%\fancyhead[R]{Bibliographie}
%\input{./biblio.tex}
	% et enfin le contenu	
	\fancyhead{}
	\fancyhead[R]{\rightmark}
	\pagenumbering{arabic}
	
	
\pagebreak

\vfill
	
\vfill
\pagebreak
	
%	\input{biblio.tex}
\end{document}

%\usepackage{geometry}
%\geometry{hmargin=1.5cm, vmargin=1.5cm}

%\geometry{scale=0.8, nohead}

%\newenvironment{changemargin}[0]{\begin{list}{}{
	%\setmarginsrb{1.5cm}{1.5cm}{1.5cm}{1.5cm}{2.84cm}{2cm}{2.84cm}{2cm}
%}\item }{\end{list}}
%Idée de développement partir de Blancarde, pour arrivée à l'IVPico, ou l'inverse à voir. A l'oral ce serait certainement mieux, tellement Blancarde version installation est simple par rapport à l'autre.