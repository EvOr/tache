Internet, network of networks, is an interconnection of autonomous systems, independent networks connected together. Representing the Internet is a very difficult task due to its size.

The aim of this project was to realize a tool to display and analyse the Internet topology. The ultimate goal was to create an evolutive piece of software to make implementation of new displaying and analysing methods easy. These purposes hide a more pedagogical one : developping skills in the use of tools we used to develop the program.

Designed with a Model-View-Controler pattern, the program is based on the boost graph library to realize topology building and computing operations. The graphical user interface is realized with Qt and is independent from the computational part to make futur development easier.

That is why the work is parallelisable : one person may work on the graphical user interface and the other one on the core layer. Merging the different parts is easy with the use of a version control system called git.

The topology is displayed with a circular rendering and come with a zoom functionnality and numerous other filtering and computing possibilities in order to allow the user to focus on one particular aspect of the graph.

Finally, the program is evolutive, performant and its development allowed us to improve our skills about the technologies involved.\\

key words : boost, Qt, graphical user interface, topology, git

