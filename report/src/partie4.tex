\section{Développement du logiciel}
\subsection{Décodeurs d'entrée}
\subsection{Développement avec boost}
La prise en main de \textit{Boost} a été la principale difficulté du projet. En effet, la documentation de \textit{Boost}, celle fournie avec la librairie et disponible sur le site officiel, ne ressemble en rien aux documentations de la stl ou de java que nous connaissions jusqu'alors...

Pour trouver les méthodes relatives aux objets, ou comment utiliser des nouvelles fonctions comme l'algorithme permettant de placer les points sur un plan (Kamada Kawai Spring Layout), il faut parcourir plusieurs pages de documentation expliquant comment créer les structures adéquates, comment appeler la fonction, avec quels arguments et quelles structures en paramètres... Cette documentation n'est pas exhaustive pour autant et manquent cruellement d'exemple. 

De plus, les erreurs retournées par gcc à la compilation de la librairie sont incompréhensibles, c'est pourquoi nous utilisons un compilateur particulier, \verb|gfilt|, compilateur utilisant perl pour analyser les erreurs de compilation et afficher des erreurs intelligibles. 





