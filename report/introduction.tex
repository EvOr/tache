\section*{Introduction}

\subparagraph{}
Au d\'ebut utilis\'e pour relier entre eux les principaux sites militaires des \'Etats-Unis, puis les universit\'es, Internet relie aujourd'hui des millions de foyers de part le monde. Il en r\'esulte un r\'eseau tr\`es dense dont la structure complexe ne peut \^etre repr\'esent\'ee d'une fa\c con id\'eale. En effet, de part sa taille, une repr\'esentation graphique de l'Internet repr\'esente un challenge scientifique.
\par
Ce projet, r\'ealis\'e dans le cadre de notre troisi\`eme ann\'ee d'\'etude \`a l'ISIMA, a pour but de d\'evelopper un programme permettant de repr\'esenter et d'analyser la topologie de l'Internet. R\'ealis\'e sur une dur\'ee de 6 mois, le rendu final doit \^etre \'evolutif de fa\c con \`a permettre le test de plusieurs modes de repr\'esentation.
Il se base sur des outils fournis par la Recherche Op\'erationnelle et la Th\'eorie des graphes ainsi que sur deux librairies sp\'ecifiques que sont : \boost et Qt, librairies qu'il s'agissait d'appr\'ehender.
\par
Apr\`es un passage en revu des outils d'analyse de la topologie de l'Internet, nous verrons les outils utilis\'es pour le d\'eveloppement du programme. Ensuite, nous nous pencherons sur son utilisation et sur l'exp\'erience utilisateur. Enfin, nous terminerons par pr\'esenter les grandes lignes de son d\'eveloppement.

