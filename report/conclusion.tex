\section*{Conclusion}

\par
L'\'etude des m\'ethodes d'analyse de la topologie de l'Internet nous a conduit \`a des exemples de repr\'esentations et \'a des outils d'analyse que nous avons int\'egr\'es au programme final.
La  disponibilit\'es des outils dans la librairie \boost a fortement influenc\'e nos choix en mati\`ere de visualisation.

En effet, la documentation souvent difficile de \boost nous a dans un premier temps ralentit car elle ne ressemblait \`a rien de ce qu'on avait connu jusque l\`a. Progressivement, nous avons surmont\'e ces difficult\'es pour arriver \`a int\'egrer une grande partie des fonctionnalit\'es voulues au logiciel.

Au final, le programme permet un affichage circulaire de la topologie avec une fonction de zoom permettant de cibler un AS en particulier quand l'utilisateur le juge n\'ecessaire. Un certain nombre de fonctions de filtrage de l'affichage sont aussi disponibles selon les donn\'ees fournies en entr\'ee.

La volont\'e d'\'evolutivit\'e du r\'esultat \`a \'et\'e au coeur des r\'eflexions sur sa structure, et l'adoption d'un mod\`ele MVC est notre r\'eponse \`a cette derni\`ere. Celui-ci doit permettre un impl\'ementation facile de nouvelles fonctions tant au niveau des couches basses que de l'interface graphique.

Tout au long de ce projet, la difficult\'e de repr\'esenter la topologie de l'Internet ne nous a pas \'echapp\'e. Il existe de nombreuse source de donn\'ees, mais il n'y a pas syst\'ematiquement d'outil graphique les accompagnant ni de m\'ethoe id\'eale pour faire cette repr\'esentation. Il s'agit donc l\`a de faire un pas dans la recherche de la bonne image.