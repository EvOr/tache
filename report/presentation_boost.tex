%\documentclass[a4paper,12pt]{article}

%\begin{document}

% \subsection{La librairie \boost}

\subparagraph{}
La librairie \boost  est un ensemble de biblioth\`eques logicielles C++ dont le but est de compléter et de parfaire les fonctionnalités du langage et de la librairie stl. La librairie \boost est compos\'ee de nombreux modules parmis lesquels on retrouve des fonctionnalit\'es de bases comme les entr\'ees / sorties ou la gestion des erreurs, mais aussi d'autres plus avanc\'ees comme des fonctions de gestion de graphes, de programmation lin\'eaire, de g\'en\'eration de nombres pseudo-al\'eatoires, de traitement d'images ou de multithreading.
\par
L'\'ecriture de nouveaux \'el\'ements pour la librairie \boost est soumise \`a un  comit\'e de lecture et l'ensemble des biblioth\`eques est plac\'e sous une licence sp\'eciale : la \textit{Boost Software Licence} qui permet d'inclure des modules \boost aussi bien dans des projets Open Source que dans des projets propri\'etaires.
\par
\'Etant donn\'e que nombre des membres du comit\'e de lecture font aussi partie du comit\'e du standard C++, plusieurs biblioth\`eques issues de la \boost ont \'et\'e s\'electionn\'ees pour faire partie de la prochaine norme du C++.
\par
Au cours de notre projet nous avons beaucoup utilis\'e la partie gestion de graphes de \boost ainsi que certaines des fonctions d'entr\'ees/sorties am\'elior\'ees.
%\end{document}
