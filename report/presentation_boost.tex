%\documentclass[a4paper,12pt]{article}

%\begin{document}

\subsection{La librairie \boost}

\par
La \boost est un ensemble de biblioth\`eques logicielles \'ecrites en C++ pour am\'eliorer les fonctionnalit\'es de celui-ci. La \boost est compos\'ee de nombreux modules dont voici quelques exemples parmis lesquels on retrouve des fonctionnalit\'es de bases comme les entr\'es / sorties, la gestion des erreurs, et aussi d'autres plus avanc\'ees comme des fonctions de gestions de graphes, de programmation lin\'eaire, de g\'en\'eration de nombres pseudo-al\'atoires, de traitement d'images, de multithreading, et bien d'autre encore.
\par
L'\'ecriture de nouveaux \'el\'ements pour la \boost est soumis \`a un  comit\'e de lecture et l'ensemble des biblioth\`eques est plac\'e sous une licence sp\'eciale : la \textit{Boost Software Licence} qui permet d'inclure des modules de \boost aussi bien dans des projets Open Source que dans des projets propri\'etaires.
\par
\'Etant donn\'e que nombre des membres du comit\'e de lecture de la \boost font partie du comit\'e du standard C++, plusieurs biblioth\`eques issues de la \boost ont \'et\'e s\'electionn\'ees pour faire partie de la prochaine norme du C++.
\par
Au cours de notre projet nous avons beaucoup utilis\'e la partie gestion de graphes de \boost ainsi que certaines des fonctions d'entr\'es / sorties am\'elior\'ees.

%\end{document}
